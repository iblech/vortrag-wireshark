\documentclass[12pt,compress]{beamer}
\usepackage{url}
\usepackage{ucs}
\usepackage[utf8x]{inputenc}
\usepackage[ngerman]{babel}
\usepackage{tabto}
\usepackage[normalem]{ulem}

\title{Netzwerksniffing mit Wireshark}
\author[Ingo Blechschmidt]{Ingo Blechschmidt \\ \ \\ Augsburger Linux-Infotag \\ 24. März 2012}
\date[2012-03-24]{\begin{minipage}{8cm}\begin{block}{Zum Mitmachen\ldots}\ldots
mit WLAN \emph{WIRESHARK} verbinden!\end{block}\end{minipage}}

\usetheme{Antibes}
\useoutertheme{split}
\usecolortheme{orchid}
\usefonttheme{serif}
\useinnertheme{rectangles}
\usepackage{palatino}
\setbeamercovered{invisible}
\setbeamertemplate{navigation symbols}{}
\setbeamertemplate{frametitle}[default][center]

\newcommand{\floatbox}[3]{%
  \raisebox{0pt}[0pt][0pt]{%
    \begin{picture}(0,0)(#1,#2)#3\end{picture}\leavevmode%
  }%
}

\newenvironment{changemargin}[2]{% 
  \begin{list}{}{% 
    \setlength{\topsep}{0pt}% 
    \setlength{\leftmargin}{#1}% 
    \setlength{\rightmargin}{#2}% 
    \setlength{\listparindent}{\parindent}% 
    \setlength{\itemindent}{\parindent}% 
    \setlength{\parsep}{\parskip}% 
  }% 
  \item[]}{\end{list}} 

\newcommand{\itembull}{{\usebeamercolor[fg]{itemize item}\usebeamertemplate{itemize item}}}

\linespread{1.1}

\begin{document}

\frame{\floatbox{-240}{7}{\includegraphics[scale=0.25]{images/flosse.png}}\titlepage}

% Netzwerksniffing mit Wireshark
% 
% Wireshark ist ein freies Werkzeug, mit dem man aus- und eingehenden
% Netzwerkverkehr mitschneiden und analysieren kann. Das ist allgemein nützlich,
% wenn man wissen möchte, wie Daten genau aussehen, die Programme, Webseiten,
% Smartphones und sonstige Geräte übers Netzwerk schicken und empfangen.
% 
% Konkret kann man Verbindungsprobleme diagnostizieren, Netzwerkprotokolle
% verstehen, das eigene Sicherheitsbewusstsein schärfen oder herausfinden, welche
% Informationen Programme beim nach Hause telefonieren übertragen.
% 
% Der Vortrag wird in die Bedienung von Wireshark einführen und dann mehrere
% Beispiele zeigen und erklären.
% 
% Ideen:
% * Timing von Google, Microsoft usw.
% * Preloading bei Chrome
% * MTU!

% Gliederung:
% 1. Grundlagen
%    Was ist Wireshark, wie funktioniert es?
%    Legalität
% 2. Live-Beispiele
%    Preloading bei Chrome
%    ARP-Poisoning, Website-Login

\appendix
\section{Bildquellen}
\frame[t]{\frametitle{Bildquellen}
  \tiny
  \begin{itemize}
    \item \url{http://smaportal.files.wordpress.com/2009/05/wireshark_icon.png}
  \end{itemize}
}

\end{document}

\end{document}
