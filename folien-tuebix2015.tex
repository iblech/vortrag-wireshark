\documentclass[12pt,compress]{beamer}
\usepackage{url}
\usepackage{ucs}
\usepackage[utf8x]{inputenc}
\usepackage[ngerman]{babel}
\usepackage{tabto}
\usepackage[normalem]{ulem}

\title{Netzwerksniffing mit Wireshark}
\author[Ingo Blechschmidt]{Ingo Blechschmidt \\[0.5em] Tübinger Linuxtag \\ 13. Juni 2015}
\date[2014-06-13]{\begin{minipage}{8cm}\begin{block}{Zum Mitmachen\ldots}\ldots
mit WLAN \emph{Wireshark} verbinden!\end{block}\end{minipage}}

\usetheme{Antibes}
\useoutertheme{split}
\usecolortheme{orchid}
\usefonttheme{serif}
\useinnertheme{rectangles}
\usepackage{palatino}
\setbeamercovered{invisible}
\setbeamertemplate{navigation symbols}{}
\setbeamertemplate{frametitle}[default][center]

\newcommand*\oldmacroB{}%
\let\oldmacroB\insertshorttitle%
\renewcommand*\insertshorttitle{%
	\oldmacroB\hfill%
  \insertframenumber\,/\,\inserttotalframenumber\hfill}

\newcommand{\floatbox}[3]{%
  \raisebox{0pt}[0pt][0pt]{%
    \begin{picture}(0,0)(#1,#2)#3\end{picture}\leavevmode%
  }%
}

\newenvironment{changemargin}[2]{% 
  \begin{list}{}{% 
    \setlength{\topsep}{0pt}% 
    \setlength{\leftmargin}{#1}% 
    \setlength{\rightmargin}{#2}% 
    \setlength{\listparindent}{\parindent}% 
    \setlength{\itemindent}{\parindent}% 
    \setlength{\parsep}{\parskip}% 
  }% 
  \item[]}{\end{list}} 

\newcommand{\itembull}{{\usebeamercolor[fg]{itemize item}\usebeamertemplate{itemize item}}}

\linespread{1.1}

\begin{document}

\frame{\floatbox{-240}{10}{\includegraphics[scale=0.25]{images/flosse.png}}\titlepage}

% Netzwerksniffing mit Wireshark
%
% Wireshark ist ein freies Werkzeug, mit dem man aus- und eingehenden
% Netzwerkverkehr mitschneiden und analysieren kann. Das ist allgemein nützlich,
% wenn man wissen möchte, wie Daten genau aussehen, die Programme, Webseiten,
% Smartphones und sonstige Geräte übers Netzwerk schicken und empfangen.
%
% Konkret kann man Verbindungsprobleme diagnostizieren, Netzwerkprotokolle
% verstehen, das eigene Sicherheitsbewusstsein schärfen oder herausfinden, welche
% Informationen Programme beim nach Hause telefonieren übertragen.
%
% Der Vortrag wird in die Bedienung von Wireshark einführen und dann mehrere
% Beispiele zeigen und erklären.
%
% Ideen:
% * Timing von Google, Microsoft usw.
% * Preloading bei Chrome
% * MTU!

% Gliederung:
% 1. Grundlagen
%    Was ist Wireshark, wie funktioniert es?
%    Motivation
%    Legalität
%    Alternativen: tcpdump, ???
% 2. Live-Beispiele
%    Preloading bei Chrome
%    ARP-Poisoning, Website-Login

\section{Grundlagen}
\subsection{Netzwerksniffing}
\frame[t]{\frametitle{Mitschneiden von Netzwerkverkehr}
  \begin{beamercolorbox}[sep=1em,wd=\textwidth]{block body}
    \textbf{Leitfrage:} \\
    Wie sehen die Daten aus, die mein Rechner verschickt und empfängt?
  \end{beamercolorbox}
  \vfill

  Wozu das?
  \begin{itemize}
    \item Um Netzwerkprotokolle zu verstehen.
    \item Um Verbindungsprobleme zu diagnostizieren.
    \item Um das eigene Sicherheitsbewusstsein zu schärfen.
    \item Um verborgene Hintergrundkommunikation
          aufzudecken.
  \end{itemize}

  \floatbox{-280}{-100}{\includegraphics[scale=0.2]{images/sniffing.jpeg}}
}

\subsection{Über Wireshark}
\frame[t]{\frametitle{Über Wireshark}
  \begin{itemize}
    \item Wireshark: freies Werkzeug zum Netzwerksniffing
    \item erste Veröffentlichung 1998 durch Gerald Combs
  \end{itemize}

  \begin{itemize}
    \item \texttt{\$ apt-get install wireshark}
    \item keine Magie -- Daten eh im Klartext vorhanden
    \item Vorsicht: Sicherheitsprobleme in Wireshark selbst
  \end{itemize}

  \begin{itemize}
    \item Alternativen: tcpdump; \\ begrenzt auch Firefox, Chrome
  \end{itemize}

  \vfill
  \floatbox{-260}{-22}{\begin{tabular}{@{}l@{}}\includegraphics[scale=0.65]{images/gerald_combs.png}\\[-0.4em]\footnotesize Gerald Combs\end{tabular}}
}

\subsection{Netzwerkarchitektur}
\frame[t]{\frametitle{Netzwerkarchitektur}
  \begin{itemize}
    \item Versand und Empfang von Netzwerkdaten in einzelnen Paketen
    \item typische Maximalgröße: 1500 Bytes bei Ethernet
    \item Adressierungsarten:
          \begin{tabbing}
            global: \= MAC-Adressen, \= \kill
            global: \> Domainnamen,  \> etwa \texttt{luga.de} \\
            global: \> IP-Adressen,  \> etwa \texttt{193.99.144.80} \\
            lokal:  \> MAC-Adressen, \> etwa \texttt{00:16:76:7d:00:c2}
          \end{tabbing}
  \end{itemize}
}

\section{Live-Vorführung}
\frame[t]{\frametitle{Live-Vorführung}
  \begin{enumerate}
    \item Erste Schritte: Ping \\
          Demonstration von ARP, DNS und ICMP
    \vfill
    \item Start von Google Chrome \\
          DNS-Prefetching, DNS-Tests
    \vfill
    \item Webseitenaufruf, Pre-Rendering
    \vfill
    \item unverschlüsselter Login
    \vfill
    \item ARP-Spoofing (Debian-Paket dsniff)
  \end{enumerate}
}

\appendix
\section{Bildquellen}
\frame[t]{\frametitle{Bildquellen}
  \tiny
  \begin{itemize}
    \item \url{http://smaportal.files.wordpress.com/2009/05/wireshark_icon.png}
    \item \url{http://www.techiwarehouse.com/userfiles/sniffing2.jpg}
    \item \url{http://www.soldierx.com/system/files/hdb/gerald_combs.png}
  \end{itemize}
}

\end{document}

\end{document}
